\documentclass[ignorenonframetext,]{beamer}
\usetheme{CambridgeUS}
\setbeamertemplate{caption}[numbered]
\setbeamertemplate{caption label separator}{:}
\setbeamercolor{caption name}{fg=normal text.fg}
\usepackage{amssymb,amsmath}
\usepackage{ifxetex,ifluatex}
\usepackage{fixltx2e} % provides \textsubscript
\usepackage{lmodern}
\ifxetex
  \usepackage{fontspec,xltxtra,xunicode}
  \defaultfontfeatures{Mapping=tex-text,Scale=MatchLowercase}
  \newcommand{\euro}{€}
\else
  \ifluatex
    \usepackage{fontspec}
    \defaultfontfeatures{Mapping=tex-text,Scale=MatchLowercase}
    \newcommand{\euro}{€}
  \else
    \usepackage[T1]{fontenc}
    \usepackage[utf8]{inputenc}
      \fi
\fi
% use upquote if available, for straight quotes in verbatim environments
\IfFileExists{upquote.sty}{\usepackage{upquote}}{}
% use microtype if available
\IfFileExists{microtype.sty}{\usepackage{microtype}}{}
\usepackage{graphicx}
\makeatletter
\def\maxwidth{\ifdim\Gin@nat@width>\linewidth\linewidth\else\Gin@nat@width\fi}
\def\maxheight{\ifdim\Gin@nat@height>\textheight0.8\textheight\else\Gin@nat@height\fi}
\makeatother
% Scale images if necessary, so that they will not overflow the page
% margins by default, and it is still possible to overwrite the defaults
% using explicit options in \includegraphics[width, height, ...]{}
\setkeys{Gin}{width=\maxwidth,height=\maxheight,keepaspectratio}

% Comment these out if you don't want a slide with just the
% part/section/subsection/subsubsection title:
%   \AtBeginPart{
%     \let\insertpartnumber\relax
%     \let\partname\relax
%     \frame{\partpage}
%   }
%   \AtBeginSection{
%     \let\insertsectionnumber\relax
%     \let\sectionname\relax
%     \frame{\sectionpage}
%   }
%   \AtBeginSubsection{
%     \let\insertsubsectionnumber\relax
%     \let\subsectionname\relax
%     \frame{\subsectionpage}
%   }
%   
%   \setlength{\parindent}{0pt}
%   \setlength{\parskip}{6pt plus 2pt minus 1pt}
%   \setlength{\emergencystretch}{3em}  % prevent overfull lines
%   \setcounter{secnumdepth}{0}
%   
%   \date{}

\title[Reproducibility in Brain Imaging]{The question of reproducibility in brain imaging}
\author[JB Poline]{Jean-Baptiste Poline \\ \texttt{jbpoline@berkeley.edu}}
\date{April 1, 2015}
\institute[UC Berkeley]{Henry Wheeler Brain Imaging Center, \\Helen Wills Neuroscience Institute, UC Berkeley, CA}

\begin{document}

\frame{\titlepage }


\begin{frame}{Outline}

\begin{itemize}[<+->]
\itemsep1pt\parskip0pt\parsep0pt
\item
  Why I came to work on reproducibility issues
\item
  Agreeing on the scale of the problem and why the situation is locked
\item
  What can \{researchers,editors,funding agencies\} do?
\item
  Conclusion: how to we change a culture.
\end{itemize}

\end{frame}

\section{Part I: Introduction}\label{part-i-introduction}

\begin{frame}{Why I came to work on this}

\begin{itemize}[<+->]
\itemsep1pt\parskip0pt\parsep0pt
\item
  Engineering background, applied mathematics to biology
\item
  Method development for brain imaging, managed research projects, large
  imaging genetics databasing. Constant interactions with cognitive
  neuroscience and clinical research.
\item
  Realized research done was poorly reproducible, inefficient and
  mistakes were ubiquitous
\item
  Many of us were playing the publication game
\item
  Invested some times in neuroinformatics, resigned from tenure job,
  took a job at Berkeley where reproducibility was considered an
  important axis.
\end{itemize}

\end{frame}

\begin{frame}{The scale of the problem in Brain Imaging and what is the
current culture}

\begin{itemize}[<+->]
\itemsep1pt\parskip0pt\parsep0pt
\item
  NIH is investing about 400M\$ each year in neuroimaging studies.
\item
  Q: ``Would the finding for which this paper is published be replicated
  on another dataset?''
\item
  Difficult to assess, only a few relevant papers, tend to show little
  replication.
\item
  20 years of experience with researchers using neuroimaging as a tool:

  \begin{itemize}[<+->]
  \itemsep1pt\parskip0pt\parsep0pt
  \item
    Phds / postdoc / tenured researchers need papers, and grants
  \item
    ``data squeezing'', p-hacking, double dipping
  \item
    Percentage of solid and replicable studies in brain mapping :
    {[}0-25\%{]}.
  \end{itemize}
\item
  Percentage of re-usable data \& analyses? Less than a few percents if
  not funded specifically for sharing.
\end{itemize}

\end{frame}

\begin{frame}{Why is the current culture very difficult to change?}

\begin{itemize}[<+->]
\itemsep1pt\parskip0pt\parsep0pt
\item
  Researchers need/want jobs, they optimize their behaviour to get jobs
\item
  Universities need money and they will optimize criteria to get grants
\item
  Conclusion: optimize publication, the widely accepted currency in
  research

  \begin{itemize}[<+->]
  \itemsep1pt\parskip0pt\parsep0pt
  \item
    research has become a competitive game : be the quickest,
    collaborate only if you need
  \item
    keep ``your'' data for yourself, or trade it for authorship
  \item
    the situation is fostering secrecy and sloppiness, no time to train
    yourself properly
  \end{itemize}
\item
  Administrations loves quantitative criteria: numbers are easier to
  manage
\item
  A great deal of cynicism / desabusement in some communities
\end{itemize}

\end{frame}

\section{Part II: Who can do what?}\label{part-ii-who-can-do-what}

\begin{frame}{Incentives and Punish by reward}

\begin{itemize}[<+->]
\item
  Most of us think we need to change the incentives, or ``punish'' bad
  behaviour
\item
  \begin{figure}[htbp]
  \centering
  \includegraphics{./img/punish_by_reward_50pc.jpg}
  \caption{Punish by reward}
  \end{figure}
\end{itemize}

\end{frame}

\begin{frame}{Do not change the incentives}

\begin{itemize}[<+->]
\itemsep1pt\parskip0pt\parsep0pt
\item
  Because these new incentives will be optimized and abused.
\item
  Because working for specific incentives is likely to be detrimental to
  the work
\item
  Because evaluation cannot be reduced to a set of numbers: it takes
  time and competences.
\item
  carrots and sticks seem to work very well for some pets, but do they
  for researchers?
\end{itemize}

\end{frame}

\begin{frame}{It is mostly the funding agencies'}

\begin{itemize}[<+->]
\itemsep1pt\parskip0pt\parsep0pt
\item
  other ``stakeholders'' are stuck in local minima, or need to feed
  their kids
\item
  they are the ones with the less constraints
\item
  consider all ethical aspects in clinical research: strong rational for
  sharing
\item
  consider how much is re-used of what has been funded : make research
  less costly
\item
  consider funding infrastructures for sharing data, train
\end{itemize}

\end{frame}

\begin{frame}{practical steps:}

\begin{itemize}[<+->]
\itemsep1pt\parskip0pt\parsep0pt
\item
  grant review guidelines have a new section: significance, feasability,
  \ldots{} \emph{replicability+re-usability},
\item
  evaluation of the applicants will include ways previous research can
  be re-used
\item
  code and data are released with appropriate license and reviewed as
  deliverable (data paper?)
\end{itemize}

\end{frame}

\begin{frame}{Universities's call: Training on statistical and
computational methods}

\begin{itemize}[<+->]
\itemsep1pt\parskip0pt\parsep0pt
\item
  train on open and collaborative science
\item
  train life science scientists in depth programming skills (``data
  science'')
\item
  train life science scientists in depth in statistics (``data
  science'')
\end{itemize}

\end{frame}

\begin{frame}{Researcher's call: The research manifesto / Hippocratic
oaths}

\begin{itemize}[<+->]
\itemsep1pt\parskip0pt\parsep0pt
\item
  I will make my research tools re-usable by others
\item
  I will strive to collaborate in the areas where I cannot do the best
  work
\item
  I will be as transparent as possible
\item
  I will take the time to train myself in key areas
\item
  I will consider first the actions benefitting the progress of
  knowledge
\end{itemize}

\end{frame}

\begin{frame}{On the statistical / computational side of things}

\begin{itemize}[<+->]
\itemsep1pt\parskip0pt\parsep0pt
\item
  The damaging effect of p-values (see psychology journal ban on
  p-values)
\item
  Effect sizes and power: journals' call
\item
  Publish models and model comparisons
\item
  Replication with open data
\item
  Code review and testing (``un-tested code is broken code'')
\end{itemize}

\end{frame}

\section{Part III: Conclusion}\label{part-iii-conclusion}

\begin{frame}{Conclusion: What shall we do about it}

\begin{itemize}[<+->]
\itemsep1pt\parskip0pt\parsep0pt
\item
  Train the new generation of scientist in computation, statistics
\item
  NIH answers:

  \begin{itemize}[<+->]
  \itemsep1pt\parskip0pt\parsep0pt
  \item
    Data Discovery Index, checklists
  \item
    online forum for open discussions,
  \item
    change in funding and bio, anonymize peer review, etc.
  \end{itemize}
\item
  Work with journals and editors to accept well powered null findings
\item
  OSF, many lab and community based projects, METRICS (Meta-Research)
  institutes
\item
  Reward people who produce re-usable science
\end{itemize}

\end{frame}

\begin{frame}{Acknowledgement}

\begin{itemize}
\itemsep1pt\parskip0pt\parsep0pt
\item
  At Berkeley: M. Brett, J. Millman, F. Perez; Simpace interest group:
  D. Sheltraw, C. Gallen, A. Tambini, K. K. Hwang; B. Inglis, M.
  D'Esposito.
\item
  At INCF: Mathew Abrams, Linda Layon, Roman Valls
\item
  Nidash: David Kennedy, Satra Ghosh, Chris Gorgowleski, Nolan Nichols,
  Dave Keator, Camille Maumet, Guillaume Flandin, Tom Nichols, Russ
  Poldrack, and others
\item
  At Pasteur: Roberto Toro; At Neurospin: B. Thirion; at the MNI: PJT
  and colleagues
\end{itemize}

\end{frame}

\begin{frame}{References}

\begin{itemize}
\itemsep1pt\parskip0pt\parsep0pt
\item
  K. Button et al., Nature Neuroscience, 2013
\item
  Nature, ``Reducing our irreproducibility'', 2013.

  \begin{itemize}
  \itemsep1pt\parskip0pt\parsep0pt
  \item
    New mechanism for independently replicating needed
  \item
    Easy to misinterpret artefacts as biologically important
  \item
    Too many sloppy mistakes
  \end{itemize}
\item
  NIH plans to enhance reproducibility. Collins and Tabak, Nature, 2014.
\item
  Boekel, W., et al. (Cortex 2013) : replication study of structural
  brain-behavior correlations.
\item
  J. Carp Cogn Affect Behav Neurosci, 2013
\item
  Poldrack, R.A., and Poline, J.-B. (2014). {[}\ldots{}{]}
  reproducibility challenges of shared data. TICS.
\item
  Griffiths, T.L. (2015). Manifesto for a new (computational) cognitive
  revolution. Cognition 135, 21--23.
\end{itemize}

\end{frame}

%---- 
\end{document}
