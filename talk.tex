\documentclass[ignorenonframetext,]{beamer}
\usetheme{CambridgeUS}
\setbeamertemplate{caption}[numbered]
\setbeamertemplate{caption label separator}{:}
\setbeamercolor{caption name}{fg=normal text.fg}
\usepackage{amssymb,amsmath}
\usepackage{ifxetex,ifluatex}
\usepackage{fixltx2e} % provides \textsubscript
\usepackage{lmodern}
\ifxetex
  \usepackage{fontspec,xltxtra,xunicode}
  \defaultfontfeatures{Mapping=tex-text,Scale=MatchLowercase}
  \newcommand{\euro}{€}
\else
  \ifluatex
    \usepackage{fontspec}
    \defaultfontfeatures{Mapping=tex-text,Scale=MatchLowercase}
    \newcommand{\euro}{€}
  \else
    \usepackage[T1]{fontenc}
    \usepackage[utf8]{inputenc}
      \fi
\fi
% use upquote if available, for straight quotes in verbatim environments
\IfFileExists{upquote.sty}{\usepackage{upquote}}{}
% use microtype if available
\IfFileExists{microtype.sty}{\usepackage{microtype}}{}
\usepackage{graphicx}
\makeatletter
\def\maxwidth{\ifdim\Gin@nat@width>\linewidth\linewidth\else\Gin@nat@width\fi}
\def\maxheight{\ifdim\Gin@nat@height>\textheight0.8\textheight\else\Gin@nat@height\fi}
\makeatother
% Scale images if necessary, so that they will not overflow the page
% margins by default, and it is still possible to overwrite the defaults
% using explicit options in \includegraphics[width, height, ...]{}
\setkeys{Gin}{width=\maxwidth,height=\maxheight,keepaspectratio}

% Comment these out if you don't want a slide with just the
% part/section/subsection/subsubsection title:
%   \AtBeginPart{
%     \let\insertpartnumber\relax
%     \let\partname\relax
%     \frame{\partpage}
%   }
%   \AtBeginSection{
%     \let\insertsectionnumber\relax
%     \let\sectionname\relax
%     \frame{\sectionpage}
%   }
%   \AtBeginSubsection{
%     \let\insertsubsectionnumber\relax
%     \let\subsectionname\relax
%     \frame{\subsectionpage}
%   }
%   
%   \setlength{\parindent}{0pt}
%   \setlength{\parskip}{6pt plus 2pt minus 1pt}
%   \setlength{\emergencystretch}{3em}  % prevent overfull lines
%   \setcounter{secnumdepth}{0}
%   
%   \date{}

\title{The question of reproducibility in brain imaging}
\author[JB Poline]{Jean-Baptiste Poline \\ \texttt{jbpoline@berkeley.edu}}
\date{April 1, 2015}
\institute[UC Berkeley]{Henry Wheeler Brain Imaging Center, \\Helen Wills Neuroscience Institute, UC Berkeley, CA}

\begin{document}

\frame{\titlepage }


\begin{frame}{The question of reproducibility in brain imaging}

\begin{block}{Jean-Baptiste Poline}

Brain Imaging Center, Helen Wills Neuroscience Institute, UC Berkeley

\end{block}

\end{frame}

\begin{frame}{Outline}

\begin{itemize}[<+->]
\itemsep1pt\parskip0pt\parsep0pt
\item
  Evidence for a crisis in reproducibility\\
\item
  What about brain imaging ?
\item
  Causes\\
\item
  Impact
\item
  What shall we do about it
\end{itemize}

\begin{itemize}[<+->]
\itemsep1pt\parskip0pt\parsep0pt
\item
  Mesh terms ``reproducibility of results'' (100 in 2010)
  \includegraphics{./img/mesh_term_reproducibility_results.pdf}
\end{itemize}

\end{frame}

\begin{frame}{Science finding Reproducibility crisis evidence}

\begin{block}{Preclinical oncology}

\begin{itemize}[<+->]
\itemsep1pt\parskip0pt\parsep0pt
\item
  Begley C.G. \& Ellis L. Nature, (2012): ``Only 6 out of 53 key
  findings in pre-clinical oncology could be fully replicated''
\end{itemize}

\end{block}

\begin{block}{Epidemiology}

\begin{itemize}[<+->]
\itemsep1pt\parskip0pt\parsep0pt
\item
  Reproducible Epidemiologic Research, Peng et al, 2006.
\item
  ``High FP/FN Ratio in Epidemiologic Studies'', Ioannidis, 2011.
\end{itemize}

\end{block}

\begin{block}{Genetics}

\begin{itemize}[<+->]
\itemsep1pt\parskip0pt\parsep0pt
\item
  Ionannidis 2007: 16 SNPs hypothesized, check on 12-32k cancer/control:
  ``\ldots{} results are largely null.''
\item
  The failing concept of endophenotype (Iacono, Psychophysiology, 2014)
\item
  Many references and warnings: eg:``Drinking from the fire hose
  \ldots{}'' by Hunter and Kraft, 2007.
\end{itemize}

\end{block}

\end{frame}

\begin{frame}

\begin{block}{In social sciences, psychology, cognitive neuroscience}

\begin{itemize}[<+->]
\itemsep1pt\parskip0pt\parsep0pt
\item
  Reproducibility Project in Psychology (Open Science Framework)
\item
  Simonshon et al. ``P-curve'', ``Evaluating Replication Results'' 2014,
  2013.
\item
  Special Issue on ``Reliability and Replication in Cognitive and
  Affective Neuroscience Research.'' Barch, Deanna, and Yarkoni, Cogn
  Affect Behav Neurosci, 2013.
\end{itemize}

\end{block}

\begin{block}{Neuroscience}

\begin{itemize}[<+->]
\itemsep1pt\parskip0pt\parsep0pt
\item
  Button et al., Nature Neuroscience, 2013
\end{itemize}

\end{block}

\begin{block}{In general: Editorials in high profile journals}

\begin{itemize}[<+->]
\itemsep1pt\parskip0pt\parsep0pt
\item
  Nature, ``Reducing our irreproducibility'', 2013.

  \begin{itemize}[<+->]
  \itemsep1pt\parskip0pt\parsep0pt
  \item
    New mechanism for independently replicating needed
  \item
    Easy to misinterpret artefacts as biologically important
  \item
    Too many sloppy mistakes
  \end{itemize}
\item
  NIH plans to enhance reproducibility. Collins and Tabak, Nature, 2014.
\end{itemize}

\end{block}

\end{frame}

\begin{frame}{What about brain imaging ? Some - \emph{but few} - facts}

\begin{itemize}[<+->]
\itemsep1pt\parskip0pt\parsep0pt
\item
  Publication does not allow replication or to find methodological
  issues (J. Carp Cogn Affect Behav Neurosci, 2013):
\end{itemize}

``For example, while Brown and Braver (2005) claimed that activation in
the anterior cingulate cortex (ACC) is sensitive to the likelihood of
committing an error, Nieuwenhuis, Tanja, Mars, Botvinick, and Hajcak
(2007) reported no relationship between ACC activation and error
likelihood.''

\begin{itemize}[<+->]
\itemsep1pt\parskip0pt\parsep0pt
\item
  When attempted, replication is poor:

  \begin{itemize}[<+->]
  \itemsep1pt\parskip0pt\parsep0pt
  \item
    Boekel, W., et al. (Cortex 2013) : replication study of structural
    brain-behavior correlations.
  \item
    5 studies, 17 findings: Bayesian analysis favored null hypothesis
  \item
    But: only 36 subjects, while most original studies were better
    powered
  \end{itemize}
\end{itemize}

\begin{itemize}[<+->]
\itemsep1pt\parskip0pt\parsep0pt
\item
  Autism example: Toro et al., Corpus callosum size example. S.
  Bookheimer's examples (cereb. size, FFA, FC).
\end{itemize}

\note{\begin{itemize}
\itemsep1pt\parskip0pt\parsep0pt
\item
  Analysis of large databases showing low concordance of small sample
  group analysis (Thirion et al., 2007)
\end{itemize}

R Toro: We conducted a meta-analysis of the literature which suggested a
statistically significant difference. However, the studies included were
heavily underpowered: on average only 20\% power to detect differences
of 0.3 standard deviations, which makes it difficult to establish the
reality of such a difference. We therefore studied the size of the
corpus callosum among 694 subjects (328 patients, 366 controls) from the
Abide cohort. Despite having achieved 99\% power to detect statistically
significant differences of 0.3 standard deviations, we did not observe
any.}

\end{frame}

\begin{frame}{Causes and Impact}

\begin{block}{Statistical}

\end{block}

\begin{block}{Computational}

\end{block}

\begin{block}{Social}

\end{block}

\end{frame}

\begin{frame}

\begin{block}{Statistical causes:(1)}

\begin{itemize}[<+->]
\itemsep1pt\parskip0pt\parsep0pt
\item
  Lack of understanding of statistical issues and power computation
\item
  The usual issues:

  \begin{itemize}[<+->]
  \itemsep1pt\parskip0pt\parsep0pt
  \item
    low power studies (Button et al, 2013)
  \item
    P-hacking: Simmons et al. 2011, Simmonshon et al., 2014
  \end{itemize}
\end{itemize}

\end{block}

\begin{block}{Example}

\begin{itemize}[<+->]
\itemsep1pt\parskip0pt\parsep0pt
\item
  From imaging genetics (BDNF - Hippocampus volume):
\item
  \includegraphics{./img/molendijk_2012_f4.pdf}
\end{itemize}

\end{block}

\end{frame}

\begin{frame}[fragile]

\begin{block}{Statistical causes:(2)}

\begin{itemize}[<+->]
\itemsep1pt\parskip0pt\parsep0pt
\item
  P value evil: will we eventually turn to Bayesian evidence? How ?
\item
  No good understanding of the necessity to report null results
  -\textgreater{} File drawer problem (Rosenthal, 1979)\\
\item
  Emmergence of complex H0/H1, A. Afraz, 2014.
\end{itemize}

\end{block}

\begin{block}{And if we stick to p-values:}

\begin{itemize}[<+->]
\itemsep1pt\parskip0pt\parsep0pt
\item
  Which one to pick: Revised standard for statistical evidence (PNAS
  Johnson 2013) : 0.05 \textless{}=\textgreater{} BF\(\in{[3,5]}\)
\item
  \includegraphics{./img/Johnson_significance.pdf}
\end{itemize}

\note{\begin{verbatim}
For instance, H1 “neurons in brain area X encode visual memories” can be
contrasted against H0 “neurons in brain area X does not encode visual
memories”. Even if H1 is not true, if only positive results get published,
after a few years there will be plenty of published papers showing that
neurons in brain area X encode visual memories. Then, some scientists might
hypothesize a more complex H1: “neurons in area X encode visual memories by
synchronization of their electrical activity”, now the null hypothesis
would be “neurons in area X encode visual memories without synchronization
of their activity”.  In this new null hypothesis, involvement of area X
neurons in memory encoding is already assumed. This second null hypothesis
can be falsely rejected again in contrast to an even more complex
scientific model, creating an even more complex set of default beliefs and
null hypotheses. 
\end{verbatim}
}

\end{block}

\end{frame}

\begin{frame}

\begin{block}{Computational causes:}

\begin{itemize}[<+->]
\itemsep1pt\parskip0pt\parsep0pt
\item
  Biologists and MDs are rarely well trained in computation - but most
  brain imaging findings rely heavily on computations
\item
  Claerbout's ``''``An article about computational science in a
  scientific publication is not the scholarship itself, it is merely
  advertising of the scholarship. The actual scholarship is the complete
  software development environment and the complete set of instructions
  which generated the figures.''``''"
\item
  Meta data capture and curation not implemented (parameters and process
  of data generation), no standards for meta data
\item
  Computational environment packaging not used (Neurodebian VM, Docker,
  \ldots{})
\end{itemize}

\end{block}

\end{frame}

\begin{frame}

\begin{block}{Social/systemic causes}

\begin{itemize}[<+->]
\itemsep1pt\parskip0pt\parsep0pt
\item
  Publication based reward system (career, grants, fame, etc) + hyper
  competitive environment is in general not working towards good
  science:

  \begin{itemize}[<+->]
  \itemsep1pt\parskip0pt\parsep0pt
  \item
    favors speed over careful, \textbf{re-usable}, reproduced studies
  \item
    favors high risk and rapid publications
  \item
    impeeds data and code sharing even for publicly funded research
  \end{itemize}
\item
  Positive finding publication bias and the file drawer problem

  \begin{itemize}[<+->]
  \itemsep1pt\parskip0pt\parsep0pt
  \item
    how this can delay scientific revolution (A. Afraz, `We could all be
    astronomers')
  \item
    is science always self-correcting ?
  \end{itemize}
\end{itemize}

\end{block}

\end{frame}

\begin{frame}

\begin{block}{Impact}

\begin{itemize}[<+->]
\itemsep1pt\parskip0pt\parsep0pt
\item
  Large amount of resource wasted (talent, money, time)
\item
  Discredit from the public and governments
\item
  Slows down scientific and medical progress
\item
  Impact on the type of work that can be started (counter example: UK
  biobank, Bavarian cohorts).
\item
  The system may select the most ``productive'' scientists - not
  necessarily the best
\end{itemize}

\end{block}

\end{frame}

\begin{frame}{Conclusion: What shall we do about it}

\begin{itemize}[<+->]
\itemsep1pt\parskip0pt\parsep0pt
\item
  Adopt more stringent and better statistical and computational
  standards
\item
  Adopt genetic research standards for replication
\item
  Adopt clinical trial standards and pre-registration
\item
  Augment the awareness of these issues, adopt data and code sharing as
  the standard in our field
\end{itemize}

\end{frame}

\begin{frame}{Conclusion: What shall we do about it}

\begin{itemize}[<+->]
\itemsep1pt\parskip0pt\parsep0pt
\item
  Train the new generation of scientist in computation, statistics
\item
  NIH answers:

  \begin{itemize}[<+->]
  \itemsep1pt\parskip0pt\parsep0pt
  \item
    Data Discovery Index, checklists
  \item
    online forum for open discussions,
  \item
    change in funding and bio, anonymize peer review, etc.
  \end{itemize}
\item
  Work with journals and editors to accept well powered null findings
\item
  OSF, many lab and community based projects, METRICS (Meta-Research)
  institutes
\item
  Reward people who produce re-usable science
\end{itemize}

\end{frame}

\begin{frame}{Acknowledgement}

\begin{itemize}
\itemsep1pt\parskip0pt\parsep0pt
\item
  At Berkeley: M. Brett, J. Millman, F. Perez; Simpace interest group:
  D. Sheltraw, C. Gallen, A. Tambini, K. K. Hwang; B. Inglis, M.
  D'Esposito.
\item
  At INCF: Mathew Abrams, Linda Layon, Roman Valls
\item
  Nidash: David Kennedy, Satra Ghosh, Chris Gorgowleski, Nolan Nichols,
  Dave Keator, Camille Maumet, Guillaume Flandin, Tom Nichols, Russ
  Poldrack, etc
\item
  At Pasteur, Neurospin, MNI.
\end{itemize}

\end{frame}

%---- 
\end{document}
